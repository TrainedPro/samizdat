\documentclass[12pt, a4paper]{report}

% Packages
\usepackage[utf8]{inputenc}
\usepackage{graphicx}
\usepackage{geometry}
\usepackage{hyperref}
\usepackage{listings}
\usepackage{xcolor}
\usepackage{float}
\usepackage[acronym]{glossaries}
\usepackage{titlesec}
\usepackage{fancyhdr}
\usepackage{setspace}
% \usepackage{biblatex} % Removed due to missing package
% \addbibresource{references.bib}

% Geometry
\geometry{
  top=0.6in,
  bottom=0.8in,
  left=1in,
  right=1in,
  includehead
}

% Header/Footer
\pagestyle{fancy}
\fancyhf{}
\rhead{\thepage}
\lhead{\leftmark}
\setlength{\headheight}{15pt}

% Title Page Data
\title{ResilientP2PTestbed: Off-Grid Disaster Communication System}
\author{Hassaan}
\date{\today}

\begin{document}

\begin{titlepage}
    \centering
    \vspace*{1cm}
    
    \Huge
    \textbf{ResilientP2PTestbed}
    
    \vspace{0.5cm}
    \LARGE
    An Off-Grid Disaster Recovery Communication System
    
    \vspace{1.5cm}
    \textbf{Hassaan Anwar (22P-9160)}\\
    \textbf{Muhammad Aais Rabbani (22P-9164)}
    
    \vspace{1cm}
    
    \textbf{Supervisor: Dr. Ali Sayyed}
    
    \vfill
    
    Department of Computer Science\\
    National University of Computer and Emerging Sciences\\
    Peshawar Campus
    
    \vspace{0.8cm}
    
    \Large
    \today
    
\end{titlepage}

\begin{abstract}
    In the wake of natural disasters, cellular infrastructure often fails, leaving affected populations and first responders without a reliable means of communication. This project, \textit{ResilientP2PTestbed}, presents a robust, off-grid mobile communication system leveraging the Google Nearby Connections API. The system utilizes a mesh network topology to facilitate text and audio communication between devices without reliance on centralized infrastructure (Internet, Cellular, Satellites). Key features include automatic peer discovery, multi-hop routing, and delay-tolerant message delivery. This report details the design, implementation, and testing of the system, demonstrating its viability as a rapid-deployment communication tool for emergency scenarios.
\end{abstract}

\tableofcontents
\listoffigures
%\listoftables

\chapter{Introduction}
\section{Project Vision}
The reliability of communication networks is often taken for granted until a catastrophic event occurs. Natural disasters such as earthquakes, floods, and hurricanes frequently damage cellular towers and fiber-optic cables, rendering traditional communication infrastructure useless. In such scenarios, the ability to coordinate rescue efforts, locate survivors, and distribute aid is severely hampered gap.

\textbf{ResilientP2PTestbed} is designed to bridge this gap. By transforming standard Android smartphones into nodes of a self-sustaining mesh network, the system enables communication in completely off-grid environments. The vision is to empower communities and first responders with a "deploy-anywhere" communication tool that requires no pre-existing infrastructure, Internet connection, or SIM card.

\section{Problem Statement}
Current disaster response protocols rely heavily on satellite phones (which are expensive and scarce) or deploying temporary cell towers (COWs), which takes time. During the critical "Golden Hour" immediately following a disaster, victims are often isolated. There is a lack of widely available, zero-cost, infrastructure-independent communication software that can run on consumer hardware people already possess.

\section{Proposed Solution}
The proposed solution, \textit{ResilientP2P}, is an Android application that leverages the Google Nearby Connections API to establish high-bandwidth peer-to-peer (P2P) connections. The system uses a \textbf{P2P\_CLUSTER} strategy to form a dynamic mesh network.

Key Objectives:
\begin{itemize}
    \item \textbf{Off-Grid}: Zero reliance on Internet/Cellular.
    \item \textbf{High Bandwidth}: Utilizes Wi-Fi Direct for voice and data.
    \item \textbf{Self-Healing}: Dynamic routing adapts to node movement.
    \item \textbf{Metric-Driven}: Includes built-in instrumentation for RSSI monitoring and network performance logging.
\end{itemize}

This project focuses on the "Phase 1" milestones: establishing a robust transport layer, ensuring reliability with application-level ACKs, and validating performance through rigorous field testing.


\chapter{Literature Review}
\section{Off-Grid Communication}
Off-grid communication has been a subject of research for decades. Traditional approaches include packet radio (AX.25) and specialized hardware like LoRaWAN mesh networks (e.g., Meshtastic). While effective, these require specialized hardware dongles that the average disaster victim does not carry.

\section{Google Nearby Connections API}
Google Nearby Connections provides a high-level abstraction over Wi-Fi Direct, Bluetooth Classic, and BLE \cite{nearbyconnections}. It handles the complexity of basic radio negotiation, making it an ideal foundation for rapid prototyping.

The API offers three connection strategies:
\begin{itemize}
    \item \textbf{P2P\_POINT\_TO\_POINT}: 1-to-1 connection, high bandwidth.
    \item \textbf{P2P\_STAR}: 1-to-N connection, hub-and-spoke.
    \item \textbf{P2P\_CLUSTER}: M-to-N connection. This strategy allows any device to connect to multiple other devices, forming a loose mesh topology. ResilientP2P primarily utilizes this strategy to support multi-hop routing.
\end{itemize}

\section{Mobile Ad-Hoc Networks (MANETs)}
Mobile Ad-Hoc Networks (MANETs) are decentralized wireless networks where each node participates in routing by forwarding data for other nodes.
Key challenges in MANETs include:
\begin{enumerate}
    \item \textbf{Dynamic Topology}: Nodes move freely, breaking links frequently.
    \item \textbf{Battery Constraints}: Mobile devices have limited power; continuous routing drains energy.
    \item \textbf{Routing Overhead}: Protocols must balance between maintaining valid routes and flooding the network with control packets. 
\end{enumerate}

Traditional protocols like \textbf{AODV} (Ad hoc On-Demand Distance Vector) and \textbf{DSR} (Dynamic Source Routing) are standard in this domain \cite{royer1999}. However, for small-scale disaster scenarios with high mobility, simple \textbf{Flooding} or \textbf{Gossip} protocols often provide better robustness despite higher redundancy.


\chapter{System Analysis}
\section{Functional Requirements}
\begin{itemize}
    \item \textbf{FR1 - Peer Discovery}: The system must automatically discover other devices running the application within radio range.
    \item \textbf{FR2 - Text Communication}: Users must be able to send text messages to direct and multi-hop peers.
    \item \textbf{FR3 - Audio Streaming}: The system must support Push-to-Talk (PTT) voice communication.
    \item \textbf{FR4 - Mesh Routing}: Messages must be forwarded to nodes not directly connected to the sender.
    \item \textbf{FR5 - Resilience}: The network must self-heal when nodes join or leave.
\end{itemize}

\section{Non-Functional Requirements}
\begin{itemize}
    \item \textbf{NFR1 - Latency}: Voice transmission latency should be minimal (< 500ms) for usability.
    \item \textbf{NFR2 - Battery Efficiency}: The application should handle power management to prolong operation during emergencies.
    \item \textbf{NFR3 - Usability}: The UI must be simple and high-contrast for use in stressful environments.
\end{itemize}

\section{Use Case Analysis}
The primary actors are \textbf{Survivors} and \textbf{Rescuers}.
\begin{itemize}
    \item \textbf{Broadcast Alert}: A survivor broadcasts an "SOS" message. It propagates through the mesh to a Rescuer.
    \item \textbf{Voice Coordination}: Rescuers use PTT to coordinate searching a building without needing line-of-sight.
    \item \textbf{Mesh Formation}: Devices automatically discover and connect to neighbors to extend network range.
\end{itemize}

Figure \ref{fig:usecase} illustrates the high-level interactions between these actors and the system.

\begin{figure}[H]
    \centering
    \includegraphics[width=0.9\textwidth]{../../../../diagrams/output/use_case_diagram.png}
    \caption{Use Case Diagram}
    \label{fig:usecase}
\end{figure}

\section{Behavioral Modeling}
To understand the dynamic behavior of the system, we analyze the sequence of events during connection establishment and data transmission.

\subsection{Sequence Diagram}
Figure \ref{fig:sequence} details the interaction flow between the UI, P2P Manager, and the Nearby Connections API when a user initiates a connection.

\begin{figure}[H]
    \centering
    \includegraphics[width=0.8\textwidth]{../../../../diagrams/output/sequence_diagram.pdf}
    \caption{Sequence Diagram: Connection Establishment}
    \label{fig:sequence}
\end{figure}

\subsection{Activity Diagram}
The overall workflow for a node participating in the mesh is depicted in Figure \ref{fig:activity}.

\begin{figure}[H]
    \centering
    \includegraphics[width=0.5\textwidth]{../../../../diagrams/output/activity_diagram.pdf}
    \caption{Activity Diagram: Mesh Participation}
    \label{fig:activity}
\end{figure}


\chapter{System Design}
\section{System Architecture}
The system follows a layered architecture, separating the UI, Management Logic, and Data Transport layers.

\subsection{Architectural Flow}
The interaction between different threads and components is visualized in the Swimlane Diagram (Figure \ref{fig:swimlane}).

\begin{figure}[H]
    \centering
    \includegraphics[width=0.9\textwidth]{../../diagrams/output/swimlane_diagram.pdf}
    \caption{Swimlane Diagram: Component Interaction}
    \label{fig:swimlane}
\end{figure}

\section{Database Design}
The application uses a local SQLite database (via Room Persistence Library) to store logs and queue persistent packets. 
Figure \ref{fig:erd} shows the Entity Relationship Diagram.

\begin{figure}[H]
    \centering
    \includegraphics[width=0.9\textwidth]{../../diagrams/output/database_schema.pdf}
    \caption{Entity Relationship Diagram (ERD) of the Local Database}
    \label{fig:erd}
\end{figure}

\section{Class Structure}
The core logic resides in the `managers` package.
\begin{itemize}
    \item \textbf{P2PManager}: Handles discovery, connection lifecycle, and routing.
    \item \textbf{HeartbeatManager}: Manages keep-alive signals and zombie detection.
    \item \textbf{VoiceManager}: Handles audio recording and playback.
\end{itemize}

Figure \ref{fig:class_diagram} illustrates the comprehensive class structure and relationships.

\begin{figure}[H]
    \centering
    \includegraphics[width=\textwidth]{../../diagrams/class_structure.pdf}
    \caption{Class Diagram of ResilientP2PTestbed}
    \label{fig:class_diagram}
\end{figure}


\chapter{Implementation Details}
\section{Mesh Routing Algorithm}
The system has evolved from simple flooding to a \textbf{Metric-Based Routing} protocol. While it still uses flooding for discovery, it maintains a routing table optimized for stability.

\subsection{Metric-Based Scoring}
Routes are evaluated based on a score derived from the Time-To-Live (TTL). A higher TTL indicates fewer hops, resulting in a higher score.
\begin{equation}
    Score = TTL
\end{equation}
When a packet arrives, the router compares the new score with the existing route's score. The routing table is updated only if the new route is strictly better ($NewScore > CurrentScore$), preventing route flapping.

\subsection{Self-Poisoning Prevention}
To prevent loops where a node learns a route to itself via a neighbor, the system inspects the `sourceId` of incoming packets. If `packet.sourceId == localId`, the packet is dropped immediately, and the route is not updated.

\begin{lstlisting}[language=Java, caption=Route Update Logic]
if (packet.sourceId != localUsername) {
    val newScore = packet.ttl
    if (newScore > currentScore) {
        routingTable.put(packet.sourceId, sourceEndpointId)
        routingScores[packet.sourceId] = newScore
    }
}
\end{lstlisting}

\section{State Management}
The application state is managed by a reactive `P2PState` data class, exposed via `StateFlow`. This ensures UI consistency across the Composable hierarchy.

Key state fields include:
\begin{itemize}
    \item \textbf{isHybridMode}: Toggles between pure P2P and Gateway-assisted modes.
    \item \textbf{isLowPower}: Reduces discovery frequency to save battery.
    \item \textbf{connectedEndpoints}: List of currently active physical connections.
    \item \textbf{logs}: A rolling buffer of the last 100 log entries for on-device debugging.
\end{itemize}

\section{Audio Streaming}
Audio is captured using `AudioRecord` at 16kHz sample rate (MONO). The raw PCM data is chunked into payloads and transmitted via the high-bandwidth Wi-Fi Direct channel. The system supports both Unicast (1-to-1) and Broadcast (1-to-All) streaming.

\section{Heartbeat Mechanism}
To detect "zombie" connections, the `HeartbeatManager` sends a ping every 5 seconds. If no packet is received for 15 seconds (reduced from 30s), the peer is marked as stale. Direct PINGs are used to verify physical link integrity independent of the mesh overlay.


\chapter{Results and Testing}
\section{Performance Testing}
\begin{itemize}
    \item \textbf{Connection Time}: Average connection establishment time is 3-5 seconds.
    \item \textbf{Range}: Effective range is approximately 30-50 meters indoors and 80-100 meters outdoors (line of sight).
    \item \textbf{Audio Latency}: Measured latency is ~200ms on single hop and ~600ms on double hop.
\end{itemize}

\section{Stress Testing}
Creating a cluster of 4 devices resulted in stable mesh performace. However, audio broadcasting to all 3 peers simultaneously caused minor frame drops, indicating bandwidth saturation limits of the current implementation.

\section{User Interface}
The UI is built with Jetpack Compose, ensuring a responsive and modern experience. The dashboard provides real-time visualization of mesh neighbors and routing activities.


\chapter{Conclusion and Future Work}
\section{Conclusion}
The \textit{ResilientP2PTestbed} successfully demonstrates the viability of smartphone-based off-grid mesh networks for disaster recovery. By abstracting the complexities of radio management through the Nearby Connections API, we achieved a stable text and voice communication platform that operates without any external infrastructure.

\section{Limitations}
\begin{itemize}
    \item \textbf{Battery Drain}: Continuous advertising and scanning consumes significant power.
    \item \textbf{Hardware Constraints}: Wi-Fi Direct range is limited by the physical antenna of the device.
\end{itemize}

\section{Future Work}
Based on the FYP2 Implementation Roadmap, the next phases of development will focus on:

\begin{enumerate}
    \item \textbf{The Gateway Bridge (Horizon 1 Extension)}:
    \begin{itemize}
        \item Implementing an Internet Detection Module to continuously monitor connectivity.
        \item Developing a Bridging Protocol to relay messages between the mesh and a central server via MQTT or WebSockets when internet is available.
    \end{itemize}
    
    \item \textbf{Store-and-Forward Engine (Resilience)}:
    \begin{itemize}
        \item Ensuring no message is lost by persisting packets in a `MessageTable`.
        \item Implementing background `WorkManager` tasks to retry delivery when neighbors reappear.
    \end{itemize}
    
    \item \textbf{Intelligent Routing (Horizon 2)}:
    \begin{itemize}
        \item Moving beyond flooding to a metric-based routing algorithm ($Cost = f(RSSI, Battery, Hops)$).
        \item Experimenting with On-Device ML (TFLite) to predict link stability.
    \end{itemize}
    
    \item \textbf{Censorship Resistance}:
    \begin{itemize}
        \item Wrapping traffic in standard HTTPS/WebSocket frames to evade DPI (Deep Packet Inspection).
        \item Supporting user-configurable self-hosted relays.
    \end{itemize}
\end{enumerate}


\bibliographystyle{plain}
\bibliography{references}

\end{document}
