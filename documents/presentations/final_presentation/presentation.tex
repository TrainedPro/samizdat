\documentclass[aspectratio=169]{beamer}

\mode<presentation>{
  \usetheme{Madrid}
  \usecolortheme{dolphin}
  \usefonttheme{professionalfonts}
  \setbeamertemplate{navigation symbols}{}
  \setbeamertemplate{caption}[numbered]
}

% Packages
\usepackage[utf8]{inputenc}
\usepackage[T1]{fontenc}
\usepackage{graphicx}
\usepackage{booktabs}
\usepackage{caption}
\usepackage{xcolor}
\usepackage[absolute,overlay]{textpos}

% --- LAYOUT & LOGO ---
\setbeamersize{text margin left=8mm,text margin right=8mm}
\addtobeamertemplate{frametitle}{}{%
  \begin{textblock*}{1cm}(1.025\textwidth,0.16cm)
    \includegraphics[height=1cm]{logo.png}
  \end{textblock*}
}

% Title Info
\title[Resilient Mobile Communication]{ResilientP2PTestbed: Off-Grid Disaster Communication System}
\author[Hassaan \& Aais]{Hassaan Anwar (22P-9160) \\ Muhammad Aais Rabbani (22P-9164)}
\institute[FAST-NUCES]{Supervised by Dr. Ali Sayyed \\ Department of Computer Science, FAST-NUCES Peshawar}
\date{\today}

\begin{document}

% Title Frame
\begin{frame}[plain]
  \vfill
  \centering
  {\LARGE\bfseries ResilientP2PTestbed}\\[6pt]
  {\large\bfseries An Off-Grid Disaster Recovery Communication System}\vspace{0.4em}
  
  \rule{0.55\textwidth}{0.6pt}\vspace{0.7em}

  {\textbf{Supervisor: Dr. Ali Sayyed}}\vspace{0.5em}

  {\large\parbox{0.85\textwidth}{\centering
     Hassaan Anwar (22P-9160)\\[0.3em]
     Muhammad Aais Rabbani (22P-9164)
  }}\vspace{0.7em}

  {\small Department of Computer Science \\ FAST-NUCES, Peshawar}\vspace{0.8em}

  \includegraphics[height=1.5cm]{logo.png}\vspace{0.3em}

  {\footnotesize Date: \today}
  \vfill
\end{frame}

\begin{frame}{Outline}
    \tableofcontents
\end{frame}

\section{Introduction}
\begin{frame}{Problem Statement}
    \begin{itemize}
        \item \textbf{Scenario:} Natural disasters (earthquakes, floods) destroy critical infrastructure.
        \item \textbf{Impact:} "Golden Hour" rescue efforts are hampered by lack of communication.
        \item \textbf{Gap:} Satellite phones are scarce; existing consumer apps lack robust mesh routing.
    \end{itemize}
\end{frame}

\begin{frame}{Proposed Solution}
    \begin{itemize}
        \item \textbf{ResilientP2P}: Smartphone-based mesh network.
        \item \textbf{Core Tech}: Google Nearby Connections (Wi-Fi Direct + BLE).
        \item \textbf{Topology}: \texttt{P2P\_CLUSTER} (Many-to-Many).
        \item \textbf{Features}:
            \begin{itemize}
                \item Zero Infrastructure Required.
                \item High Bandwidth (Voice + Data).
                \item Self-Healing Routing.
            \end{itemize}
    \end{itemize}
\end{frame}

\section{System Design}
\begin{frame}{System Architecture & Diagrams}
    The system design is captured in the following diagrams. Click to open.
    \vspace{0.5cm}
    \begin{itemize}
        \item \href{run:./diagrams/class_structure.pdf}{\beamerbutton{Class Diagram (PDF)}} - Core application structure.
        \item \href{run:./diagrams/use_case_diagram.png}{\beamerbutton{Use Case Diagram (PNG)}} - Survivor/Rescuer interactions.
        \item \href{run:./diagrams/sequence_diagram.pdf}{\beamerbutton{Sequence Diagram (PDF)}} - Connection establishment flow.
        \item \href{run:./diagrams/swimlane_diagram.pdf}{\beamerbutton{Swimlane Diagram (PDF)}} - Component interaction and threading.
        \item \href{run:./diagrams/database_schema.pdf}{\beamerbutton{Database Schema (PDF)}} - Room database ERD.
    \end{itemize}
\end{frame}

\section{Implementation}
\begin{frame}{Implementation Details}
    \begin{block}{Mesh Routing}
        Flooding protocol with TTL limits and Message ID caching to prevent broadcast loops.
    \end{block}
    
    \begin{block}{Zombie Detection}
        Heartbeats Sent every 5s. Peers disconnected after 30s inactivity.
    \end{block}
    
    \begin{block}{Audio}
        16kHz PCM Audio Chunks transmitted over Wi-Fi Direct.
    \end{block}
\end{frame}

\section{Results}
\begin{frame}{Performance Results}
    \begin{itemize}
        \item \textbf{Range}: 40m (Indoors), 100m (Outdoors).
        \item \textbf{Latency}: Voice $<$ 200ms per hop.
        \item \textbf{Stability}: Successfully maintains cluster of 4+ devices.
    \end{itemize}
\end{frame}

\section{Conclusion}
\begin{frame}{Conclusion & Future Work}
    \textbf{Conclusion}:
    A functional, high-bandwidth off-grid mesh system running on standard Android hardware.

    \textbf{Future Work}:
    \begin{itemize}
        \item End-to-End Encryption (ECDH).
        \item Optimized Routing (DSR/AODV).
        \item Integration with LoRaWAN gateways.
    \end{itemize}
\end{frame}

\begin{frame}[plain]
  \vfill
  \centering
  {\Huge\bfseries Thank You!}\\[8pt]
  \rule{0.45\textwidth}{0.6pt}\\[0.9em]

  {\large We welcome your questions and feedback.}\\[2em]
  
  \includegraphics[height=1.5cm]{logo.png}\vspace{0.3em}
  \vfill
\end{frame}

\end{document}
