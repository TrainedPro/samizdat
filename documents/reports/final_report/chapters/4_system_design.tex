\section{System Architecture}
The system follows a layered architecture, separating the UI, Management Logic, and Data Transport layers.

\subsection{Architectural Flow}
The interaction between different threads and components is visualized in the Swimlane Diagram (Figure \ref{fig:swimlane}).

\begin{figure}[H]
    \centering
    \includegraphics[width=0.9\textwidth]{../../diagrams/output/swimlane_diagram.pdf}
    \caption{Swimlane Diagram: Component Interaction}
    \label{fig:swimlane}
\end{figure}

\section{Database Design}
The application uses a local SQLite database (via Room Persistence Library) to store logs and queue persistent packets. 
Figure \ref{fig:erd} shows the Entity Relationship Diagram.

\begin{figure}[H]
    \centering
    \includegraphics[width=0.9\textwidth]{../../diagrams/output/database_schema.pdf}
    \caption{Entity Relationship Diagram (ERD) of the Local Database}
    \label{fig:erd}
\end{figure}

\section{Class Structure}
The core logic resides in the `managers` package.
\begin{itemize}
    \item \textbf{P2PManager}: Handles discovery, connection lifecycle, and routing.
    \item \textbf{HeartbeatManager}: Manages keep-alive signals and zombie detection.
    \item \textbf{VoiceManager}: Handles audio recording and playback.
\end{itemize}

Figure \ref{fig:class_diagram} illustrates the comprehensive class structure and relationships.

\begin{figure}[H]
    \centering
    \includegraphics[width=\textwidth]{../../diagrams/class_structure.pdf}
    \caption{Class Diagram of ResilientP2PTestbed}
    \label{fig:class_diagram}
\end{figure}
