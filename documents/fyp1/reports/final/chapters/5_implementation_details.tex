\section{Mesh Routing Algorithm}
The system has evolved from simple flooding to a \textbf{Metric-Based Routing} protocol. While it still uses flooding for discovery, it maintains a routing table optimized for stability.

\subsection{Metric-Based Scoring}
Routes are evaluated based on a score derived from the Time-To-Live (TTL). A higher TTL indicates fewer hops, resulting in a higher score.
\begin{equation}
    Score = TTL
\end{equation}
When a packet arrives, the router compares the new score with the existing route's score. The routing table is updated only if the new route is strictly better ($NewScore > CurrentScore$), preventing route flapping.

\subsection{Self-Poisoning Prevention}
To prevent loops where a node learns a route to itself via a neighbor, the system inspects the `sourceId` of incoming packets. If `packet.sourceId == localId`, the packet is dropped immediately, and the route is not updated.

\begin{lstlisting}[language=Java, caption=Route Update Logic]
if (packet.sourceId != localUsername) {
    val newScore = packet.ttl
    if (newScore > currentScore) {
        routingTable.put(packet.sourceId, sourceEndpointId)
        routingScores[packet.sourceId] = newScore
    }
}
\end{lstlisting}

\section{State Management}
The application state is managed by a reactive `P2PState` data class, exposed via `StateFlow`. This ensures UI consistency across the Composable hierarchy.

Key state fields include:
\begin{itemize}
    \item \textbf{isHybridMode}: Toggles between pure P2P and Gateway-assisted modes.
    \item \textbf{isLowPower}: Reduces discovery frequency to save battery.
    \item \textbf{connectedEndpoints}: List of currently active physical connections.
    \item \textbf{logs}: A rolling buffer of the last 100 log entries for on-device debugging.
\end{itemize}

\section{Audio Streaming}
Audio is captured using `AudioRecord` at 16kHz sample rate (MONO). The raw PCM data is chunked into payloads and transmitted via the high-bandwidth Wi-Fi Direct channel. The system supports both Unicast (1-to-1) and Broadcast (1-to-All) streaming.

\section{Heartbeat Mechanism}
To detect "zombie" connections, the `HeartbeatManager` sends a ping every 5 seconds. If no packet is received for 15 seconds (reduced from 30s), the peer is marked as stale. Direct PINGs are used to verify physical link integrity independent of the mesh overlay.
